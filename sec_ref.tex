\begin{titlepage}
\sectioncentered*{РЕФЕРАТ}
\label{sec:ref}

Дипломный проект предоставлен следующим образом. Электронные носители: 1
компакт-диск. Чертежный материал: 6 листов формата А1. Пояснительная записка:
113 страниц, 8 рисунков, 3 таблицы, 12 литературных источников, 3 приложения.

Ключевые слова: системные вызовы, сбор данных, процессы, файловые дескрипторы,
оптимизация.

Объектом исследования и разработки является возможность реализации системных
вызовов для получения информации о процессах и открытых файлах.

Целью данного дипломного проекта является создание интерфейса, который будет
быстрее существующей файловой системы \texttt{procfs} и не будет требовать
лишних ресурсов.

При разработке программного средства использовался язык программирования Си,
редактор Vim. 

В результате использование разработанных системных вызовов ускоряет работу
типичных программ в несколько раз, а непосредственно процесс получения
информации до нескольких десятков раз.

Областью практического применения программного средства являются
высоконагруженные системы, требовательные к производительности и низкому
потреблению памяти. Изменение работы основных утилит для использования
разработанных вызовов значительно ускорит их работу и уменьшит потребление
памяти системой, в результате чего даже при критических нагрузках мониторинг
состояния системы не будет негативно сказываться на работе системы.

Разработанный программный продукт можно считать экономически эффективным, и он
полностью оправдывает вложенные в него средства.

Дипломный проект является завершенным, поставленные задачи были решены в полной
мере, возможно дальнейшее развитие продукта путем добавления новых флагов для
существующих вызовов, а также реализации новых системных вызовов, заменяющих
другие части \texttt{procfs}.
\end{titlepage}
