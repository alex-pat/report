\section{СИСТЕМНОЕ ПРОЕКТИРОВАНИЕ}
\label{sec:sys}

Изучив теоретические аспекты разработки кода ядра, рассмотрев различные
существующие используемые и неиспользуемые аналоги и выработав список
необходимых для разработки требований, разбиваем систему на компоненты.
Компоненты в виде блоков и их взаимосвязи указаны на чертеже.
В разрабатываемом модуле можно выделить следующие блоки:
\begin{itemize}
\item системный вызов \texttt{sys\_pidmap};
\item системный вызов \texttt{sys\_pidinfo};
\item системный вызов \texttt{sys\_fdmap};
\item системный вызов \texttt{sys\_pidfdopen};
\item системный вызов \texttt{sys\_pidfdinfo};
\item процесс пространства пользователя.
\end{itemize}


\subsection{Системный вызов \texttt{sys\_pidmap}}
\label{sub:sys:sys_pidmap}


\subsection{Системный вызов \texttt{sys\_pidinfo}}
\label{sub:sys:sys_pidinfo}


\subsection{Системный вызов \texttt{sys\_fdmap}}
\label{sub:sys:sys_fdmap}


\subsection{Системный вызов \texttt{sys\_pidfdopen}}
\label{sub:sys:sys_pidfdopen}


\subsection{Системный вызов \texttt{sys\_pidfdinfo}}
\label{sub:sys:sys_pidfdinfo}
