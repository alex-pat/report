\section{СИСТЕМНОЕ ПРОЕКТИРОВАНИЕ}
\label{sec:sys}

Изучив теоретические аспекты разработки кода ядра, рассмотрев различные
существующие используемые и неиспользуемые аналоги и выработав список
необходимых для разработки требований, разбиваем систему на компоненты, что
улучшит гибкость системы
Компоненты в виде блоков и их взаимосвязи указаны на чертеже.
В разрабатываемом модуле можно выделить следующие блоки:
\begin{itemize}
\item системный вызов \texttt{sys\_pidmap};
\item системный вызов \texttt{sys\_pidinfo};
\item системный вызов \texttt{sys\_fdmap};
\item системный вызов \texttt{sys\_pidfdopen};
\item системный вызов \texttt{sys\_pidfdinfo};
\item подсистема управления задачами ядра;
\item подсистема виртульаной файловой системыы;
\item процессы пространства пользователя.
\end{itemize}

Структурная схема, которая иллюстрирует все перечисленные блоки и связи
между ними, приведена на чертеже ГУИР.400201.055 C1.

Каждый блок системы выполняет свою определенную задачу и взаимодействует с
другими модулями последством передачи различных бинарных данных. 

Процесс пространства пользователя, чтобы получить нужные данные о процессах,
делает системные вызовы \texttt{sys\_pidmap, sys\_pidinfo}.
При этом он дополнительно может передавать информацию о том, какие именно данные
ему необходимо получить, а какие ему не потребуются.
После этого системные вызовы обращаются к подсистеме управления задачами ядра,
которая позволяет им получить соответствующие данные. Затем полученные данные
системные вызовы возвращают программе в пространстве ядра. 

Для получения данных о файловых дескрипторах процессов и открытия их процесс
пространства пользователя исполняет такие системные вызовы как
\texttt{sys\_fdmap}, \texttt{sys\_pidfdinfo} и \texttt{sys\_pidfdopen},
передавая ядру информацию о том, какие данные необходимо получить, а какие ему
не потребуются. Затем системный вызов обращается к подсистеме виртуальной
файловой системы и их для получения необходимых данных и возвращает их процессу
в пространстве пользователя.

\subsection{Системный вызов \texttt{sys\_pidmap}}
\label{sub:sys:sys_pidmap}

Системный вызов \texttt{sys\_pidmap} необходим для получения информации о
виртуальной памяти процесса, такой как виртуальные адреса отображений в память
процесса, флаги доступа, размер отображений, путь к файлу (если отображение
файловое, а не анонимное), значение \texttt{RSS}~--- размер страниц памяти,
выделенных процессу операционной системой и в настоящее время находящихся в ОЗУ,
значение \texttt{swap}~--- количество страниц, перемещенных из ОЗУ во вторичное
хранилище. 

\subsection{Системный вызов \texttt{sys\_pidinfo}}
\label{sub:sys:sys_pidinfo}

Для отображений в виртуальную память существует возможность получить файловый
дескриптор для каждого отображения процесса. Для этого необходимо открыть
следующий файл:

\medskip
\begin{lstlisting}[style=cstyle]
/proc/4285/map\_files/7fe107a82000-7fe107a83000
\end{lstlisting}
\medskip
где \texttt{4285} -- идентификатор процесса,
\texttt{7fe107a82000-7fe107a83000} -- начальный и конечный виртуальный адрес
отображения соответственно.

Данное действие позволит получить нужный файловый дескриптор. Системный вызов
\texttt{sys\_pidinfo} необходим для обеспечения подобного механизма. 

\subsection{Системный вызов \texttt{sys\_fdmap}}
\label{sub:sys:sys_fdmap}

Системный вызов \texttt{mmap()} позволяет делать отображения в виртуальное
адресное пространство процесса. Такие отображения могут быть анонимные и
файловые. Для файловых отображений существует возможность получить путь файла
для каждого отображения процесса при помощи команды: 

\medskip
\begin{lstlisting}[style=cstyle]
 readlink /proc/4368/map_files/7fbe6c374000-7fbe6c378000
\end{lstlisting}
\medskip
где \texttt{4285} -- идентификатор процесса,
\texttt{7fbe6c374000-7fbe6c378000}~--- начальный и конечный виртуальный адрес
отображения соответственно. 

\subsection{Системный вызов \texttt{sys\_pidfdopen}}
\label{sub:sys:sys_pidfdopen}

Системный вызов \texttt{sys\_pidfdopen} необходим для получения информации о
привелегиях пользователей: все идентификаторы пользователя и групп, список
групп, права доступа пользователя.

\subsection{Системный вызов \texttt{sys\_pidfdinfo}}
\label{sub:sys:sys_pidfdinfo}

Файлы в директории \texttt{/proc/\$PID/fdinfo/} позволяют получать информацию
об открытых дескрипторах процесса. Такие отображения могут быть анонимные и
файловые. Для файловых отображений существует возможность получить позицию в
файле файла для каждого отображения процесса при помощи команды: 

\medskip
\begin{lstlisting}[style=cstyle]
 readlink /proc/4368/map_files/7fbe6c374000-7fbe6c378000
\end{lstlisting}
\medskip
где \texttt{1141} -- идентификатор процесса,
\texttt{7fbe6c374000-7fbe6c378000}~--- начальный и конечный виртуальный адрес
отображения соответственно. 

Системный вызов \texttt{sys\_pidfdinfo} необходим для обеспечения подобного
механизма. 

\subsection{Процессы пространства пользователя}
\label{sub:sys:sys_pidfdinfo}

Пользовательский процесс~--- это процесс, который исполняется на уровне
привилегий пользователя, и которому необходимо получать некоторые данные из
ядра, например, с целью мониторинга, сбора информации и статистических данных о
системе: данные об использовании памяти, о загрузке процессоров, о процессах,
пользователях, файлах и файловых системах, об использовании различных ресурсов,
сетевая статистика, информация о состоянии устройств в системе.

Подобного рода программа должна быть реализована наиболее оптимально, чтобы даже
на сильно загруженной системе она выполняла свои функции максимально быстро и
качественно. Для достижения этой цели данной программе необходимо делать
системные вызовы.

Настоящий дипломный проект содержит как реализацию примеров программ,
использующих реализованные возможности ядра, так и с использованием
традиционного интерфейса \texttt{procfs}. Кроме того, реализованы тесты
системных вызовов с использованием инструмента Kselftests, который размещается в
директории \texttt{tools/testing/selftests/} относительно корня исходных текстов
ядра.

