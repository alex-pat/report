\sectioncentered*{ВВЕДЕНИЕ}
\addcontentsline{toc}{section}{ВВЕДЕНИЕ}
\label{sec:intro}
На сегодняшний день важное место в жизни людей занимают высокопроизводительные
системы. Одним из аспектов функционирования таких систем является мониторинг
состояния системы, в частности, процессов и файлов. Большинство подобных систем
работают под управлением операционной системы GNU/Linux, в которой традиционным
средством для доступа к информации о системе является виртуальная файловая
система \texttt{procfs}.
Такой подход обусловлен философией UNIX, и имеет такие преимущества как
отсутствие необходимости в специальных системных вызовах и удобство чтения
человеком, однако требует для получения необходимой информации следующих
действий от разработчика:

\begin{itemize}
\item генерация строки, содержащей путь к виртуальному файлу;
\item открытие файла или директории;
\item чтение и парсинг прочитанных текстовых данных;
\item закрытие файла;
\item чтение символьной ссылки.
\end{itemize}

Генерация и парсинг строк требует временных и вычислительных затрат на переводы
бинарных данных в текстовый вид и обратно. Открытие файлов и директорий требует
значительных затрат на проход пути и выделений памяти для внутренних структур
ядра, что не только увеличивает потребление программой оперативной памяти, но и
заполняет кэши елементов директорий, негативно влияя на производительность.
Кроме того, искомая информация распределена по значительному числу файлов, что
требует частого выполнения вышеописанных действий.

Если на потребительских устройствах нет высоких требований к времени выполнения
и негативное влияние лишних действий незаметно, то в высокопроизводительных
системах важна экономия процессорного времени и памяти. Кроме того, для конечной
цели сбора информации не имеют значения побочные структуры, открытые файлы и
переводы данных в иную фуорму, важен только конечный результат: полезные
бинарные данные.

Целью данного дипломного проекта является разработка альтернативного быстрого
средства получения информации о процессах и файловых дейскрипторах, минуя лишние
действия как в пространстве пользователя, так и в пространстве ядра.
