\section{ФУНКЦИОНАЛЬНОЕ ПРОЕКТИРОВАНИЕ}
\label{sec:func}

В данном разделе описывается функционирование и интерфейс между модулями
разрабатываемой системы. В первую очередь перед этим необходимо выделить и
конкретизировать необходимые для реализации критерии, составляющие
результирующие преимущества системы, а также определить путь решения
поставленных задач. К таким вопросам относятся:

\begin{itemize}
\item формат и состав структур;
\item расположение данных переменной длины;
\item возоможность пользователю определять состав получаемых данных;
\item долговечность формата данных;
\item обход контроля пространством пользователя размера выделяемой памяти.
\end{itemize}

Опишем подробнее каждый из пунктов. Как было определено ранее и как показывает
мировая практика, оптимальным форматом данных является бинарная структура, а не
текстовый поток. Кроме того, для предотвращения получения пользовательской
программой лишних данных и структурирования интерфейса системы, необходимо
логически разделить возможные передаваемые данные на подчасти. Описанные условия
будут описаны далее при описании интерфейса каждого системного вызова.

Часто при разработке интерфейсов встаёт задача об отпимальном размещении данных,
которые могут занимать либо крайне большие объемы памяти, либо быть вообще
неограниченными по размеру. Примерами таких данных являются имена файлов и
директорий, а также полные пути к файлам. В случае с системами семейства UNIX,
стандарт POSIX ставит достаточно высокие верхние границы размера имён: 255 байт
для имени и 4095 байт для целого пути. Очевидно, что выделять такие объемы
памяти при каждом вызове, даже если результат будет занимать заведомо мало
памяти~--- неоптимальное решение. Общепринятым решением является размещение
данных переменной длины в конце структуры, при этом размер строки указывается
равным нули или единице. Примером таких данных является структура
\texttt{linux\_dirent}, использующаяся в системном вызове \texttt{getdents()}:

\medskip
\begin{adjustwidth}{\fivecharsapprox}{}
\begin{lstlisting}[basicstyle=\fontencoding{T1}\small\ttfamily]
struct linux_dirent {
	unsigned long	d_ino;
	unsigned long	d_off;
	unsigned short	d_reclen;
	char		d_name[1];
};
\end{lstlisting}
\end{adjustwidth}
\medskip

В данном случае поле \texttt{d\_name} используется для хранения имени элемента
директории. Так как в языке Си строки нуль-терминированные, нет необходимости в
дополнительном поле размера имени. Подобный подход с расположением данных
переменного размера в конце структуры используется не только в случае имён и
путей к файлам, но и в случаях отпимизации выделения строк малого размера в
структуре ядра \texttt{dentry} или стандартной библиотеке языка С++. Такое
решение позволяет сэкономить не только выделенную память, но и уменьшить число
кэш-промахов во время работы со структурами, так как непосредственно строка
располагается рядом с описывающей её структурой.

