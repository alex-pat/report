\newcommand{\rub}{руб}
\section{ТЕХНИКО-ЭКОНОМИЧЕСКОЕ ОБОСНОВАНИЕ ЭФФЕКТИВНОСТИ РАЗРАБОТКИ БИНАРНЫХ
  СИСТЕМНЫХ ВЫЗОВОВ ДЛЯ СБОРА ИНФОРМАЦИИ О ПРОЦЕССАХ И ФАЙЛОВЫХ ДЕСКРИПТОРАХ}

% Begin Calculations

\FPeval{\totalProgramSize}{9810}
\FPeval{\totalProgramSizeCorrected}{2760}

\FPeval{\normativeManDays}{200}

\FPeval{\additionalComplexity}{0.12}
\FPeval{\complexityFactor}{clip(1 + \additionalComplexity)}

\FPeval{\stdModuleUsageFactor}{0.7}
\FPeval{\originalityFactor}{0.65}

\FPeval{\adjustedManDaysExact}{clip( \normativeManDays * \complexityFactor * \stdModuleUsageFactor * \originalityFactor )}
\FPround{\adjustedManDays}{\adjustedManDaysExact}{0}

\FPeval{\daysInYear}{365}
\FPeval{\redLettersDaysInYear}{9}
\FPeval{\weekendDaysInYear}{104}
\FPeval{\vocationDaysInYear}{21}
\FPeval{\workingDaysInYear}{ clip( \daysInYear - \redLettersDaysInYear - \weekendDaysInYear - \vocationDaysInYear ) }

\FPeval{\developmentTimeMonths}{3}
\FPeval{\developmentTimeYearsExact}{clip(\developmentTimeMonths / 12)}
\FPround{\developmentTimeYears}{\developmentTimeYearsExact}{2}
\FPeval{\requiredNumberOfProgrammersExact}{ clip( \adjustedManDays / (\developmentTimeYears * \workingDaysInYear) + 0.5 ) }

% тут должно получаться 2 ))
\FPtrunc{\requiredNumberOfProgrammers}{\requiredNumberOfProgrammersExact}{0}

\FPeval{\tariffRateFirst}{50}
\FPeval{\tariffFactorFst}{3.04}
\FPeval{\tariffFactorSnd}{3.48}


\FPeval{\employmentFstExact}{clip( \adjustedManDays / \requiredNumberOfProgrammers )}
\FPtrunc{\employmentFst}{\employmentFstExact}{0}

\FPeval{\employmentSnd}{clip(\adjustedManDays - \employmentFst)}


\FPeval{\workingHoursInMonth}{160}
\FPeval{\salaryPerHourFstExact}{clip( \tariffRateFirst * \tariffFactorFst / \workingHoursInMonth )}
\FPeval{\salaryPerHourSndExact}{clip( \tariffRateFirst * \tariffFactorSnd / \workingHoursInMonth )}
\FPround{\salaryPerHourFst}{\salaryPerHourFstExact}{0}
\FPround{\salaryPerHourSnd}{\salaryPerHourSndExact}{0}

\FPeval{\bonusRate}{1.5}
\FPeval{\workingHoursInDay}{8}
\FPeval{\totalSalaryExact}{clip( \workingHoursInDay * \bonusRate * ( \salaryPerHourFst * \employmentFst + \salaryPerHourSnd * \employmentSnd ) )}
\FPround{\totalSalary}{\totalSalaryExact}{0}

\FPeval{\additionalSalaryNormative}{20}

\FPeval{\additionalSalaryExact}{clip( \totalSalary * \additionalSalaryNormative / 100 )}
\FPround{\additionalSalary}{\additionalSalaryExact}{0}

\FPeval{\socialNeedsNormative}{0.6}
\FPeval{\socialProtectionNormative}{34}
\FPeval{\socialProtectionFund}{ clip(\socialNeedsNormative + \socialProtectionNormative) }

\FPeval{\socialProtectionCostExact}{clip( (\totalSalary + \additionalSalary) * \socialProtectionFund / 100 )}
\FPround{\socialProtectionCost}{\socialProtectionCostExact}{0}

\FPeval{\taxWorkProtNormative}{4}
\FPeval{\taxWorkProtCostExact}{clip( (\totalSalary + \additionalSalary) * \taxWorkProtNormative / 100 )}
\FPround{\taxWorkProtCost}{\taxWorkProtCostExact}{0}
\FPeval{\taxWorkProtCost}{0} % это считать не нужно, зануляем чтобы не менять формулы

\FPeval{\stuffNormative}{3}
\FPeval{\stuffCostExact}{clip( \totalSalary * \stuffNormative / 100 )}
\FPeval{\stuffCost}{\stuffCostExact}

\FPeval{\timeToDebugCodeNormative}{15}
\FPeval{\reducingTimeToDebugFactor}{0.3}
\FPeval{\adjustedTimeToDebugCodeNormative}{ clip( \timeToDebugCodeNormative * \reducingTimeToDebugFactor ) }

\FPeval{\oneHourMachineTimeCost}{0.5}

\FPeval{\machineTimeCostExact}{ clip( \oneHourMachineTimeCost * \totalProgramSizeCorrected / 100 * \adjustedTimeToDebugCodeNormative ) }
\FPround{\machineTimeCost}{\machineTimeCostExact}{0}

\FPeval{\businessTripNormative}{15}
\FPeval{\businessTripCostExact}{ clip( \totalSalary * \businessTripNormative / 100 ) }
\FPround{\businessTripCost}{\businessTripCostExact}{0}

\FPeval{\otherCostNormative}{20}
\FPeval{\otherCostExact}{clip( \totalSalary * \otherCostNormative / 100 )}
\FPround{\otherCost}{\otherCostExact}{0}

\FPeval{\overheadCostNormative}{100}
\FPeval{\overallCostExact}{clip( \totalSalary * \overheadCostNormative / 100 )}
\FPround{\overheadCost}{\overallCostExact}{0}

\FPeval{\overallCost}{clip( \totalSalary + \additionalSalary + \socialProtectionCost + \taxWorkProtCost + \stuffCost + \machineTimeCost + \businessTripCost + \otherCost + \overheadCost ) }

\FPeval{\supportNormative}{30}
\FPeval{\softwareSupportCostExact}{clip( \overallCost * \supportNormative / 100 )}
\FPround{\softwareSupportCost}{\softwareSupportCostExact}{0}


\FPeval{\baseCost}{ clip( \overallCost + \softwareSupportCost ) }

\FPeval{\profitability}{35}
\FPeval{\incomeExact}{clip( \baseCost / 100 * \profitability )}
\FPround{\income}{\incomeExact}{0}

\FPeval{\estimatedPrice}{clip( \income + \baseCost )}

\FPeval{\localRepubTaxNormative}{3.9}
\FPeval{\localRepubTaxExact}{clip( \estimatedPrice * \localRepubTaxNormative / (100 - \localRepubTaxNormative) )}
\FPround{\localRepubTax}{\localRepubTaxExact}{0}
\FPeval{\localRepubTax}{0}

\FPeval{\ndsNormative}{20}
\FPeval{\ndsExact}{clip( (\estimatedPrice + \localRepubTax) / 100 * \ndsNormative )}
\FPround{\nds}{\ndsExact}{0}


\FPeval{\sellingPrice}{clip( \estimatedPrice + \localRepubTax + \nds )}

\FPeval{\taxForIncome}{0}
\FPeval{\incomeWithTaxes}{clip(\income * (1 - \taxForIncome / 100))}
\FPround\incomeWithTaxes{\incomeWithTaxes}{0}

% End Calculations

\subsection{Краткая характеристика программного продукта}

Разработанная в данном дипломном проекте система представляет собой набор
изменений в ядре Linux, добавляющих функциональность, предназначенную быть
альтернативой существующей файловой системе \texttt{procfs}, предоставляя данные
о процессах и файлах более оптимальным способом. Продукт продоставляет
возможности получать:
\begin{itemize}
\item список существующих процессов в операционной системе;
\item информацию о процессах;
\item список файловых дескрипторов;
\item информацию об открытых файлах;
\item открывать файлы процессов, даже если они были удалены, но ещё удерживаются
  программой.
\end{itemize}

При этом, программный продукт обеспечивает меньшую нагрузку на систему и более
быстрое выполнение, а также упрощает разработку программных продуктов, ранее
использовавних \texttt{procfs}. В результате этого обеспечится прирост прибыли
за счет экономии расходов на заработную плату в результате снижения трудоемкости
поддержки будущих проектов.

Само по себе ядро Linux является свободным (и бесплатным) проектом,
распространяемым с использованием стандартной общественной лицензией GNU второй
версии. Свобода программного продукта подразумевает наличие у пользователя
свободы использования в любых целях, изучения, распространения и публикации
изменений на тех же условиях. Другими словами, нет возможности получать прибыль
с продажи разработаного программного продукта. Таким образом, разработанная в
диполмном проекте система разработана для собственных нужд, однако может
использоваться коммерческими организациями в своей инфраструктуре, и экономию
такой организации рассчитать возможно.

\subsection{Расчет затрат на разработку программного продукта}

Оценка стоимости создания ПО со стороны разработчика предполагает составление
сметы затрат, которая включает следующие статьи расходов:
\begin{itemize}
\item основную заработную плату исполнителей ($ \text{З}_{\text{o}} $);
\item дополнительную заработную плату исполнителей ($\text{З}_{\text{д}} $);
\item отчисления на социальные нужды ($ \text{З}_\text{сн} $);
\item прочее (машинное время, накладные расходы, расходы на сопровождение и
  адаптацию).
\end{itemize}

Основная заработная плата исполнителей на конкретное ПО рассчитывается по
формуле:
\begin{equation}
  \label{eq:econ:total_salary}
  \text{З}_{\text{о}} = \text{K}_{\text{пр}} \cdot
                        \sum^{\text{n}}_{\text{i} = 1}
                        \text{З}_{\text{ч}}^{i} \cdot
                        \text{t}_{\text{i}}
                          \text{\,,}
\end{equation}
\begin{explanation}
где & $ \text{n} $ & количество исполнителей;\\
    & $ \text{З}_{\text{ч}}^{i} $ & часовая заработная плата \mbox{$ i $-го}
    исполнителя, \rub.$/$час; \\
    & $ \text{t}_{\text{i}} $ & трудоемкость работ, выполняемых \mbox{$ i $-ым}
    исполнителем; \\
    & $ \text{К}_{\text{пр}} $ & коэффициент премирования.
\end{explanation}

Информация о работниках перечислена в таблице~\ref{table:econ:programmers}.

\bigskip
\begin{table}[ht]
  \caption{Работники, занятые в проекте}
  \label{table:econ:programmers}
  \begin{tabular}{| >{\centering}m{0.20\textwidth}
                  | >{\centering}m{0.16\textwidth}
                  | >{\centering}m{0.12\textwidth}
                  | >{\centering}m{0.11\textwidth}
                  | >{\centering}m{0.11\textwidth}
                  | >{\centering\arraybackslash}m{0.14\textwidth}|}
   \hline
   Исполнитель & Вид работ & Месячная зарплата по тарифу, \rub. & Часовая зарплата по тарифу, \rub. & Трудо-емкость работ, ч. & Зарплата по тарифу, \rub.\\
   \hline
   Программист & Разработка & $ 600 $ & $ 3.57 $ & $ 336 $  & $ 1200 $\\
   \hline
   Ведущий программист &  Руководство  & $ 2500 $ & $ 11.9 $ & $ 168 $ & $ 2000 $\\
   \hline
    \multicolumn{5}{|c|}{ Премия, \%  } & $ 50 $\\
    \hline
    \multicolumn{5}{|c|}{ Итого на заработную плату работников, \rub.  } & $ 4800 $\\
    \hline
  \end{tabular}
\end{table}

Часовая заработная плата определяется путем деления месячной заработной платы на
количество рабочих часов в месяце. Подставив ранее вычисленные значения и даные
из таблицы таблицы~\ref{table:econ:programmers} в
формулу~(\ref{eq:econ:total_salary}) и приняв коэффициент премирования
$ \text{К}_{\text{пр}} = \num{\bonusRate} $ получим:
\begin{equation}
  \label{eq:econ:total_salary_calc}
  \text{З}_{\text{о}} = (\num{3.57} \times \num{336} + \num{11.9} \times \num{168}) \times \num{\bonusRate} = \num{4800} {\text{ \rub}} \text{\,.}
\end{equation}

Дополнительная заработная плата включает выплаты предусмотренные
законодательством от труде и определяется по нормативу в процентах от основной
заработной платы:
\begin{equation}
  \label{eq:econ:additional_salary}
  \text{З}_{\text{д}} = 
    \frac {\text{З}_{\text{о}} \cdot \text{Н}_{\text{д}}} 
          {100\%} \text{\,,}
\end{equation}
\begin{explanation}
  где & $ \text{Н}_{\text{д}} $ & норматив дополнительной заработной платы, $ \% $.
\end{explanation}

Приняв норматив дополнительной заработной платы
$ \text{Н}_{\text{д}} = \num{\additionalSalaryNormative\%} $ и подставив
известные данные в формулу~(\ref{eq:econ:additional_salary}) получим:
\begin{equation}
  \label{eq:econ:additional_salary_calc}
  \text{З}_{\text{д}} = 
    \frac{\num{4800} \times 20\%}
         {100\%} \approx \num{960}{\text{ \rub}} \text{\,.}
\end{equation}

Согласно действующему законодательству отчисления в фонд социальной защиты
населения составляют \num{\socialProtectionNormative\%} , в фонд обязательного
страхования~--- \num{\socialNeedsNormative\%}, от фонда основной и
дополнительной заработной платы исполнителей. Общие отчисления на социальную
защиту рассчитываются по формуле:
\begin{equation}
  \label{eq:econ:soc_prot}
  \text{З}_{\text{сз}} = 
    \frac{(\text{З}_{\text{о}} + \text{З}_{\text{д}}) \cdot \text{Н}_{\text{сз}}}
         {\num{100\%}} \text{\,.}
\end{equation}

Подставив вычисленные ранее значения в формулу~(\ref{eq:econ:soc_prot}) получаем
\begin{equation}
  \label{eq:econ:soc_prot_calc}
  \text{З}_{\text{сз}} =
    \frac{ (\num{4800} + \num{\additionalSalary}) \times \num{\socialProtectionFund\%} }
         { \num{100\%} }
    \approx \num{1745}{\text{ \rub}} \text{\,.}
\end{equation}

\begin{comment}
  Расчет налогов от фонда оплаты труда производится формуле
  \begin{equation}
    \label{eq:econ:tax_work_prot}
    \text{Н}_{\text{е}} = 
      \frac{(\text{З}_{\text{о}} + \text{З}_{\text{д}}) \cdot \text{Н}_{\text{не}}}
           {\num{100\%}} \text{\,,}
  \end{equation}
  \begin{explanation}
    где & $ \text{Н}_{\text{не}} $ & норматив налога, уплачиваемый единым платежом, $ \% $.
  \end{explanation}

Подставив ранее вычисленные значения в формулу~(\ref{eq:econ:tax_work_prot}) и
приняв норматив налога $ \text{Н}_{\text{не}} = \num{\taxWorkProtNormative\%} $
получаем
  \begin{equation}
    \label{eq:econ:tax_work_prot_calc}
    \text{Н}_{\text{е}} = 
        \frac{ (\num{4800} + \num{\additionalSalary}) \times \num{\taxWorkProtNormative\%} }
           { \num{100\%} }
      \approx \SI{\taxWorkProtCost}{\text{\rub}}\text{\,.}
  \end{equation}
\end{comment}

Статья расходов <<прочие затраты>> включает в себя расходы на приобретение и
подготовку специальной научно-технической  и специальной литературы. Затраты
определяются по нормативу принятому в организации в процентах от основной
заработной платы и вычисляются по формуле
\begin{equation}
  \label{eq:econ:other_cost}
  \text{П}_{\text{з}} =
    \frac{ \text{З}_{\text{о}} \cdot \text{Н}_{\text{пз}} }
         { \num{100\%} } \text{\,,}
\end{equation}
\begin{explanation}
  где & $ \text{Н}_{\text{пз}} $ & норматив прочих затрат в целом по организации, $ \% $.
\end{explanation}

Приняв значение норматива прочих затрат
$ \text{Н}_{\text{пз}} = \num{50\%} $ и подставив вычисленные
ранее значения в формулу~(\ref{eq:econ:other_cost}) получаем
\begin{equation}
  \label{eq:econ:other_cost_calc}
  \text{П}_{\text{з}} =
    \frac{ \num{4800} \times \num{50\%} }
         { \num{100\%} } = 
    \num{2400}{\text{ \rub}} \text{\,.}
\end{equation}

Общая ct рассчитывается по формуле
\begin{equation}
  \label{eq:econ:overall_cost}
  \text{С}_{\text{П}} =
    \text{З}_{\text{о}} +
    \text{З}_{\text{д}} +
    \text{З}_{\text{сз}} +
    \text{Р}_{\text{н}}\text{\,.}
\end{equation}

Подставляя ранее вычисленные значения в формулу~(\ref{eq:econ:overall_cost})
получаем

\begin{equation}
  \label{eq:econ:overall_cost_calc}
  \text{С}_{\text{П}} = \num{10147}{\text{ \rub}} \text{\,.}
\end{equation}

\subsection{Расчет экономической эффективности}

Программный продукт разрабатывается для собственных нужд. Внедрение проекта
позволит оптимизировать существующие затраты операционной системы, применяя его
в новых проектах. Кроме того, это позводит сократить время разработки будущих
проектов.

Экономия затрат на заработкую плату при использовании программного продукта в
расчете на объем выполняемых работ рассчитывается по формуле:
\begin{equation}
  \label{eq:econ:zarplt}
  \text{Э}_{\text{з}} =
  \text{K}_{\text{пр}} \cdot
  (\text{t}_{\text{c}} - \text{t}_{\text{н}}) \cdot
  \text{T}_{\text{ч}} \cdot
  \text{N}_{\text{п}} \cdot
  \text{N}_{\text{п}} \cdot
  (1 + \frac{\text{Н}_{\text{д}}}{100}) \cdot
  (1 + \frac{\text{Н}_{\text{но}}}{100}) \text{\,,}
\end{equation}
\begin{explanation}
  где & $ \text{N}_{\text{п}} $ & плановый объем работ; \\
      & $ \text{t}_{\text{c}}, \text{t}_{\text{н}} $ & трудоемкость выполнения
      работы до и после внедрения программного продукта, нормо-час; \\
      & $ \text{T}_{\text{ч}} $ & часовая тарифная ставка, соответствующая
      разряду выполняемых работ, руб./ч (3,57 руб./ч.); \\ 
      & $ \text{K}_{\text{пр}} $ & коэффициент премий (1,5); \\

      & $ \text{Н}_{\text{д}} $ & норматив дополнительной заработной платы (10\%); \\ 
      & $ \text{Н}_{\text{но}} $ & ставка отчислений от заработной платы,
      включаемых в себестоимость (34,6\%). \\
\end{explanation}

Выполнение работы, изначально занимавшей всё время рабочего часа сотрудника ($
\text{t}_{\text{c}} = 1 \text{ ч.}$) стала выполняться на 6\% быстрее, то есть $
\text{t}_{\text{н}} = 0,94 \text{ ч.}$ Экономия на заработной плате и
начислениях на заработную плату составит:
\begin{equation}
  \text{Э}_{\text{з}} = 1,5 \times (1 - 0,94) \times 3,57 \times
  1000 \times 12 \times 1,5 \times 1,346 \approx 7784 \text{ \rub} \text{\,.}
\end{equation}

Прирост чистой прибыли равен экономии на заработной плате и начислениях на
заработную плату с вычетом налога на прибыль, равного 18\%:
\begin{equation}
  \Delta\text{П}_{\text{ч}} = 7784 \times (\num{100\%} - \num{18\%}) \approx \num{6383} \text{ \rub}\text{\,.}
\end{equation}

\subsection{Расчет показателей эффективности использования программного продукта}

Для расчета показателей экономической эффективности использования данного
проекта необходимо полученные суммы прироста чистой прибыли и капитальных
вложений по годам привести к единому времени~--- расчетному году~--- путем
умножения прироста чистой прибыли и затрат за каждый год на коэффициент
привидения ALFAt, расчитываемый по формуле:
\begin{equation}
  \text{ALFA}_{\text{t}} = (1 + \text{E}_{\text{H}})^{\text{t}_\text{p}-\text{t}} \text{\,,}
\end{equation}
\begin{explanation}
  где & $ \text{E}_{\text{H}} $ & норматив приведения разновременных затрат и
  результатов, 25\%; \\
      & $ \text{t}_{\text{p}} $ & расчетный год; \\
      & $ \text{t} $ & номер года, результаты и затраты которого приводятся к
      расчетному (2018 – 1, 2019 – 2, 2020 – 3).\\ 
\end{explanation}

Подставив в формулу соответствующие значения, получим результат для следующих
трех лет:
\begin{equation}
  \text{ALFA}_{\text{1}} = (1 + 0,25)^{(1-1)} = 1 \text{\,.}
\end{equation}
\medskip
\begin{equation}
  \text{ALFA}_{\text{2}} = (1 + 0,25)^{(1-2)} = 0,8 \text{\,.}
\end{equation}
\bigskip
\begin{equation}
  \text{ALFA}_{\text{3}} = (1 + 0,25)^{(1-3)} = 0,64\text{\,.}
\end{equation}

Рассчитанная выше экономия будет достигнута за год использования программного
продукта. В компании запланировано использование системы в 2018 году, поэтому
прибыль за этот год будет в два раза меньше. 

В таблице~\ref{table:econ:calculated_data} приведены результаты расчета
показателей эффективности. Рентабельность инвестиций в разработку и внедрение
считается при помощи следующей формулы:
\begin{equation}
  \text{Р}_{\text{И}} = \frac{\text{П}_{\text{чср}} \cdot \num{100\%}}{\text{З}}\text{\,.}
\end{equation}

Среднегодовая величина чистой прибыли за расчетный период определяется по
формуле:

\begin{equation}
  \text{П}_{\text{чср}} = \frac{\sum_{i=1}^n \text{П}_{\text{чi}}}{\text{З}}\text{\,.}
\end{equation}

\begin{table}
\caption{Результат расчета показателей эффективности}
\label{table:econ:calculated_data}
  \begin{tabular}{| >{\raggedright}m{0.47\textwidth} 
                  | >{}m{0.13\textwidth} 
                  | >{}m{0.13\textwidth} 
                  | >{\arraybackslash}m{0.13\textwidth}|}
    \hline
    Показатели & 2018 & 2019 & 2020 \\

    \hline
    Прирост прибыли за счет экономии затрат & 3892 & 6383 & 6383 \\

    \hline
    Коэффициент дисконтирования & 1 & 0,8 & 0,64 \\

    \hline
    Результат с учетом фактора времени & 3245 & 6227 & 4981 \\

    \hline
    Инвестиции в разработку & 10147 & - & - \\

    \hline
    Инвестиции с учетом фактора времени & 10147 & - & - \\

    \hline
    Чистый дисконтный доход & 6902 & 6227 & 4981 \\

    \hline
    ЧДД нарастающим итогом & 6902 & 675 & 4306 \\

    \hline
  \end{tabular}
\end{table}

\begin{equation}
  \text{П}_{\text{Ч}} = \frac{3892 + 6383 + 6383}{\text{З}} = 5553 \text{ \rub}\text{\,.}
\end{equation}

Таким образом, рентабельность инвестиций в разработку данного программного
продукта будет равна:
\begin{equation}
  \text{P}_{\text{И}} = \frac{5553 \cdot 100\%}{10147} = 54\% \text{\,.}
\end{equation}

В результате технико-экономического обоснования применения программного продукта
были получены значения показателей эффективности. Полученные данные показывают,
что применение программного продукта является эффективным и инвестиции в его
разработку целесообразны.
