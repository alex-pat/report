\section{РУКОВОДСТВО ПОЛЬЗОВАТЕЛЯ}
\label{sec:manual}

Разработанные системные вызовы представляют собой набор патчей на языке Си, при
необходимости прикладываемых к ядру. Пользователь может как использовать готорые
примеры прикладных программ, так и пользоваться новыми системными вызовами
напрямую.

\subsection{Требования к аппаратному и программному обеспечению}

Минимальные требования для сборки и функционирования модифицированного ядра:
\begin{itemize}
\item GNU C 3.2;
\item GNU make 3.81;
\item binutils 2.20;
\item flex 2.5.35;
\item bison 2.0;
\item util-linux 2.10o;
\item module-init-tools 0.9.10;
\item e2fsprogs 1.41.4;
\item jfsutils 1.1.3;
\item reiserfsprogs 3.6.3;
\item xfsprogs 2.6.0;
\item squashfs-tools 4.0;
\item btrfs-progs 0.18;
\item pcmciautils 004;
\item quota-tools 3.09;
\item PPP 2.4.0;
\item isdn4k-utils 3.1pre1;
\item nfs-utils 1.0.5;
\item procps 3.2.0;
\item oprofile 0.9;
\item udev 081;
\item grub 0.93;
\item mcelog 0.6;
\item iptables 1.4.2;
\item openssl \& libcrypto 1.0.0;
\item bc 1.06.95;
\item XZ Utils 5.0;
\item tar 1.20;
\item процессор совместимый с Intel i386;
\item свободное место на жестком диске 350 Мб;
\item оперативная память 512 Mб.
\end{itemize}

\subsection{Руководство по сборке и запуску программного средства}

Перед сборкой и установной ядра необходимо получить соответствующие исходные
коды, распаковав архив с необходимой версией ядра и приложив к ней патчи:

\medskip
\begin{lstlisting}[style=cstyle]
$ tar xf linux-4.14.2.tar.xz
$ cd linux-4.14.2
$ patch -p1 < pateenok.patch
$ patch -p1 < brouka.patch
\end{lstlisting}
\medskip

При необходимости можно использовать другие версии ядра, которые доступны на
сайте \texttt{https://www.kernel.org/}. Далее необходимо проконфигурировать
ядро. Одни из наиболее простых вариантов~--- использование конфигурации текущей
системы или стандартной конфигурации текущей архитектуры. При этом полезно
создать отдельную директорию для файлов сборки:

\medskip
\begin{lstlisting}[style=cstyle]
$ mkdir ../obj/
$ cp /boot/config-$(uname -r) ../obj/.config
$ # make O=../obj defconfig
\end{lstlisting}
\medskip

Конфигурация ядра может производиться несколькими способами, при этом важно
включить параметры \texttt{CONFIG\_PIDMAP} и \texttt{CONFIG\_FDMAP}:
\begin{itemize}
\item \texttt{make O=../obj config}~--- простой текстовый интерфейс, спрашивающий о
  каждой конфигурационной опции;
\item \texttt{make O=../obj menuconfig}~--- псевдографический интерфейс на базе Ncurses;
\item \texttt{make O=../obj xconfig}~--- графический интерфейс на базе Qt;
\item \texttt{make O=../obj gconfig}~--- графический интерфейс на GTK+;
\item \texttt{make O=../obj oldconfig}~--- текстовый интерфейс, берет за основу
  конфигурацию в \texttt{.config}, спрашивая только о новых конфигурационных параметрах.
\end{itemize}

После конфигурации можно приступить к конфигурации ядра:
\medskip
\begin{lstlisting}[style=cstyle]
$ make O=../obj -j4
\end{lstlisting}
\medskip

После установки необходимо установить ядро в систему:
\medskip
\begin{lstlisting}[style=cstyle]
$ sudo make O=../obj install
$ sudo make O=../obj modules_install
\end{lstlisting}
\medskip
