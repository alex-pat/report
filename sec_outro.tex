\pagebreak
\sectioncentered*{ЗАКЛЮЧЕНИЕ}
\addcontentsline{toc}{section}{ЗАКЛЮЧЕНИЕ}
\label{sec:outro}

В результате дипломного проектирования был разработан набор бинарных системных
вызовов для получения информации о процессах и файловых дескрипторах для
операционной системы Linux, а также примеры программ пространства пользователя,
использующих новые вызовы.

В процессе проектирования реализуемой системы было выявлено, что текущий способ
получения информации о системе имеет существенные накладные расходы как по
времени, так и по потреблению памяти, что особенно заметно себя проявляет на
высоконагруженных системах. Было проведено исследование способов взаимодействия
пространства пользователя и ядра, а также существующих аналогов, в том числе, из
других операционных систем. Кроме того, был описаны возможные проблемы при
поддержке ядра в целом и системных вызовов в частности.

Разработанные вызовы позволяют получать список процессов в системе, открытых
файлов процесса, информацию о каждом процессе и файле, а также предоставяют
возможность открыть удаленный файл, если он еще не закрыт одним из процессов.
Преимуществами разработки являются высокая скорость, отсутствие лишних издержек,
возможность в ряде случаев контролировать состав получаемых данных,
однозначность, устойчивость к внутренним изменениям архитектуры ядра.

По результатам деятельности в рамках данного дипломного проектирования патчи с
разработанными системными вызовами были посланы в основную ветку разработки
ядра, из-за чего возникает вероятность принятия разработки и использования
любым желающим после обновления ядра до соответствующей версии. Практика
показывает, что по различным причинам достаточно часто патчи претерпевают
изменения и доработки до момента принятия, а также само обсуждение патчей может
происходить достаточно длительное время. Однако, даже в случае непринятия патчей
в основное ядро, желающие могут при необходимости приложить патчи к своему ядру
и использовать разработки в своих целях.

Проект разработан в полном объеме и полностью соответствует поставленной цели. В
качестве возможного направления развития разработанной системы можно
рассматривать добавление дополнительных возможностей по управлению форматом
получаемых данных, а также реализацию вызовов, заменяющих новые файлы системы
\texttt{procfs}.
