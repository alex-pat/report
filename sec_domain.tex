\section{ОБЗОР ЛИТЕРАТУРЫ}
\label{sec:domain}

\subsection{Обзор существующих аналогов}
\label{sub:domain:analogs}

Одним из интерфейсов получения информаци о процессах и потоках в операционной
системе Windows является библиотека Tool Help\cite{tool_help_article}. Ядром
библиотеки является поняние \textit{snapshot}. Snapshot~--- моментальный снимок
(копия только для чтения) одного или нискольких списков, находящихся в системной
памяти: процессы, потоки, модули и кучи.

Процессы, использующие функции Tool Help, получают доступ к этим спискам из
моментальных снимков, а не непосредственно из операционной системы.
Непосредственно сами списки в системной памяти постоянно изменяются, когда
процессы запускаются и завершаются, потоки создаются и уничтожаются, исполняемые
модули загружаются и выгружаются из системной памяти, а кучи создаются и
уничтожаются. Использование информации из снимка предотвращает несоответствия и
гонки. В противном случае изменения в списке могут привести к тому, что поток
будет неправильно перемещаться по списку или вызвать General Protection Fault.
Например, если приложение обходит список потоков, когда другие потоки создаются
или завершаются, информация, которую приложение использует для перемещения по
списку потоков, может устаревать и может вызвать ошибку для приложения,
перемещающегося по списку.

Снимок создаётся функцией \texttt{CreateToolhelp32Snapshot}. Она имеет слудующее
определение\cite{win_tool_help}:

\bigskip
\begin{adjustwidth}{\fivecharsapprox}{}
\begin{lstlisting}[language=C, basicstyle=\small\ttfamily]
HANDLE WINAPI CreateToolhelp32Snapshot(
  _In_ DWORD dwFlags,
  _In_ DWORD th32ProcessID
  );
\end{lstlisting}
\end{adjustwidth}
\bigskip

Предоставляется возможным управление содержимым снимка, указав одно или
несколько следующих констант в параметре \texttt{th32ProcessID} при вызове
функции:
\begin{itemize}
\item \texttt{TH32CS\_SNAPHEAPLIST};
\item \texttt{TH32CS\_SNAPMODULE};
\item \texttt{TH32CS\_SNAPPROCESS};
\item \texttt{TH32CS\_SNAPTHREAD}.
\end{itemize}

Флаги \texttt{TH32CS\_SNAPHEAPLIST} и \texttt{TH32CS\_SNAPMODULE} применими при
получении информации об одном процессе. Когда эти значения указаны, списки кучи
и модулей указанного процесса включаются в снимок. При передаче нуля в качестве
идентификатора процесса, используется текущий процесс. При передаче флага
\texttt{TH32CS\_SNAPTHREAD} всегда создается общесистемный снимок, даже если
передан валидный идентификатор процесса.

Для получения состояния куч или модулей всех процессов, передаётся флаг
\texttt{TH32CS\_SNAPALL} и идентификатор текущего процесса. Затем для каждого
дополнительного процесса в снимке снова необходимо вызывать функцию
\texttt{CreateToolhelp32Snapshot} с указанием его идентификатора и флаг
\texttt{TH32CS\_SNAPHEAPLIST} или \texttt{TH32CS\_SNAPMODULE}.

При прохождении по списку процессов используется внутренний итератор. Для
получения информации о первом процессе в списке, используется функция
\texttt{Process32First}. Для дальнейшего перемещения по списку процессов для
последующих записей используется функция \texttt{Process32Next}. Обе функции
заполняют структуру \texttt{PROCESSENTRY32} информацией о процессе в из снимка.
Структура \texttt{PROCESSENTRY32} имеет следующее определение:

\bigskip
\begin{adjustwidth}{\fivecharsapprox}{}
\begin{lstlisting}[language=C, basicstyle=\small\ttfamily]
typedef struct tagPROCESSENTRY32 {
  DWORD     dwSize;
  DWORD     cntUsage;
  DWORD     th32ProcessID;
  ULONG_PTR th32DefaultHeapID;
  DWORD     th32ModuleID;
  DWORD     cntThreads;
  DWORD     th32ParentProcessID;
  LONG      pcPriClassBase;
  DWORD     dwFlags;
  TCHAR     szExeFile[MAX_PATH];
} PROCESSENTRY32, *PPROCESSENTRY32;
\end{lstlisting}
\end{adjustwidth}
\bigskip

Как видно из определения, предоставленный разработчиками системы интерфейс
позволяет получить такие параметры процесса, как идентификатор, число потоков,
идентификатор родительского процесса, базовый приоритет созданных данным
просцессом потоков, имя исполняемого файла процесса. Четыре поля структуры
больше не используются, что показывает важность и серьёзность построения
интерфейса, по возможности устойчивого к изменениям архитектуры системы, при
которых элементы существующего интерфейса теряют смысл.

\subsection{Аналитический обзор}
\label{sub:domain:analitic_overview}
В источнике \cite{kernel_docs} меньше всего пунктов бибтеха.

В источнике \cite{rlove} больше всего пунктов бибтеха.
В источнике \cite{understanding} средненько пунктов бибтеха.

В источнике \cite{tanenbaum} средненько пунктов бибтеха.

В источнике \cite{johnson} средненько пунктов бибтеха.
В источнике \cite{kernelnewbies} средненько пунктов бибтеха.
В источнике \cite{vagin} средненько пунктов бибтеха.
В источнике \cite{lkml} средненько пунктов бибтеха.
В источнике \cite{anatomy} средненько пунктов бибтеха.
В источнике \cite{lwn} средненько пунктов бибтеха.
А источник \cite{map} карта!
Источник \cite{greg}~--- книга Грега.
В источнике \cite{profarch} средненько пунктов бибтеха.
В источнике \cite{man_syscall} средненько пунктов бибтеха.
